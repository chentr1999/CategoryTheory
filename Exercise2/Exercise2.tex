\documentclass{article}

\usepackage{graphicx}
\usepackage{amsmath,amssymb,mathabx,mathtools,xcolor,listings,lmodern,amsthm}   
\usepackage{url}                
\usepackage{proof}

\begin{document}
    \section*{Exercise2.1}
    \begin{proof}
        $ $\\
        (i) (epic $\Rightarrow$ surjective)

        For epic arrow $f:A\rightarrow B$ we use the proof by contradiction.
        
        Assuming that $f$ is not surjective, then there exists $b\in B$ such that for all $a\in A$, $f(a)\ne b$. 
        
        Let $i,j : B\rightarrow D$ such that $i(x)=j(x)$ for all $x\ne b\in B$ and $i(b)\ne j(b)$. Then for all $a\in A$, $if(a) = jf(a)$ but $i\ne j$. Therefore, $f$ is not epic which is a contradiction.

        Therefore $f$ is epic implies $f$ is surjective.\\
        (ii) (surjective $\Rightarrow$ epic)

        For surjective mapping $f:A\rightarrow B$. If $if = jf$, then $\forall b\in B.\; \exists a\in A.\; i(b)=if(a)=jf(a)=j(b)$. Therefore $f$ is also epic.
    \end{proof}

    
\end{document}